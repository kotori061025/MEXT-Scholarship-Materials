\documentclass{article}
\linespread{2}
\usepackage{amsmath}

\begin{document}
\noindent Q1(1):\\
$2a^2b^3\times(-3ab^2)^2\div(-6a^3b^5)\\
=\frac{18a^4b^7}{-6a^3b^5}\\
=\boxed{-3ab^2}$

\vspace{1cm}
\hrule
\vspace{1cm}

\noindent Q1(2):\\
$$|x-1|<3$$
$$-3<x-1<3$$
$$\boxed{-2}<x<\boxed4$$

\vspace{1cm}
\hrule
\vspace{1cm}

\noindent Q1(3):\\
$x+\frac{1}{x}=\frac{x^2+1}{x}=\frac{3x}{x}=\boxed3$.\\
$x^2+\frac{1}{x^2}=(x+\frac{1}{x})^2-2=3^2-2=\boxed7$.

\vspace{1cm}
\hrule
\vspace{1cm}

\noindent Q1(4):\\
i): All the three cards are odd. The probability=$\frac{C_3^5}{C_3^{10}}=\frac{10}{120}=\frac{1}{\boxed{12}}$.\\
\\
ii): There are two cases:\\
-All the three cards are even. The probability=$\frac{1}{12}$.\\
-One card is even while the other two are odd. \\
The probability=$\frac{C_1^5\cdot C_2^5}{C_3^{10}}=\frac{50}{120}=\frac{5}{12}$.\\
Given the above, the total probability=$\frac{1}{12}+\frac{5}{12}=\frac{1}{\boxed2}$.

\vspace{1cm}
\hrule
\vspace{1cm}

\noindent Q1(5):\\
As the three vectors form a triangle, we have $\vec{AB}+\vec{BC}+\vec{CA}=\vec{0}$, i.e. $<x+2,y-3,z+5>=<0,0,0>$.\\
By solving, we have $x=\boxed{-2}$, $y=\boxed3$ and $z=\boxed{-5}$.\\
$\vec{AB}\cdot\vec{AC}=<-2,2,1>\cdot<-3,5,5>=6+10+5=\boxed{21}$.

\vspace{1cm}
\hrule
\vspace{1cm}

\noindent Q1(6):\\
$y=\log_2(8x-16)=\log_2(8(x-2))=\log_28+\log_2(x-2)=3+\log_2(x-2)$.\\
Therefore, the graph shifts by $\boxed2$ on the x-axis and $\boxed3$ on the y-axis to the graph of $y=\log_2x$.

\vspace{1cm}
\hrule
\vspace{1cm}

\noindent Q1(7):\\
Let $d$ be the common different, as $2+3d=-7$, we have $d=-3$.\\
 Therefore, the arithmetic progression is $2,\boxed{-1},\boxed{-4},-7$.\\
Let $r$ be the common ratio, as $-6r^2=-54$, we have $r=\pm3$.\\
Therefore, the arithmetic progression is $\boxed{\pm2}, -6, \boxed{\pm 18}, -54$.

\vspace{1cm}
\hrule
\vspace{1cm}

\noindent Q1(8):\\
Note that $\frac{1}{k(k+1)}=\frac{1}{k}-\frac{1}{k+1}$.\\
Therefore, $\sum\limits_{k=1}^n\frac{1}{k(k+1)}=1-\frac{1}{n+1}$ by the telescoping property.\\
Then, $$1-\frac{1}{n+1}=\frac{6}{7}$$
$$n+1=7$$
$$n=\boxed6$$

\vspace{1cm}
\hrule
\vspace{1cm}

\noindent Q1(9):\\
$f^\prime(x)=(x-5)+(x+3)$ by the product rule. Therefore, $f^\prime(-1)=-6+2=\boxed{-4}$.

\vspace{1cm}
\hrule
\vspace{1cm}

\noindent Q2:\\
(1): By completing the square, we have $y=(x+\frac{1}{2})^2-\frac{9}{2}$.\\
Therefore, the vertex is $(\frac{\boxed{-1}}{2},\frac{\boxed{-9}}{2})$.\\
When a and b have common points, then the equation $x^2+x-2=3x+a$ has solution, i.e.
$$\Delta=4-4(-2-a)\geq0$$
$$a\geq\boxed{-3}$$
\\
(2): If $a=0$, then by solving $x^2+x-2=3x+1$, i.e. $x^2-2x-3=0$, we have $x=\boxed{-1},\boxed3$.\\
Therefore, the area=$\int_{-1}^3((3x+1)-(x^2+x-2))dx\\
=[-\frac{1}{3}x^3+x^2+3x]_{-1}^3\\
=-9+9+9-\frac{1}{3}-1+3\\
=\frac{\boxed{32}}{3}$.\\
When $r$ reaches its upper boundary, it is tangent to $y=3x+1$.\\
By that time, the equation $x^2+(3x+1)^2=r^2$ has only one solution., i.e.
$$\Delta=36-4(10)(1-r^2)=0$$
$$r^2=\frac{1}{10}$$
$$r=\frac{\boxed{\sqrt{10}}}{10}$$

\vspace{1cm}
\hrule
\vspace{1cm}

\noindent Q3:\\
(1): $\frac{1}{3}-(\frac{1}{5}-\frac{7}{2})=\frac{1}{3}+\frac{33}{10}=\frac{109}{30}\approx3.6\approx\boxed4$.\\
\\
(2): $(\sqrt5-\sqrt2)^2=7-2\sqrt{10}$. As $3\sqrt9<\sqrt{10}<\sqrt{10.5625}=3.25$, we have $0.5<7-2\sqrt{10}<1$. Therefore, the closest integer is \boxed{1}.\\
\\
(3): $\sin30^\circ+\cos45^\circ+\tan60^\circ=\frac{1}{2}+\frac{\sqrt2}{2}+\sqrt3\approx0.5+0.7+1.7=2.9\approx\boxed3$.\\
\\
(4): $\sum\limits_{i=1}^5(\frac{2}{3})^i=\frac{(\frac{2}{3})^6-\frac{2}{3}}{\frac{2}{3}-1}=2-\frac{2^6}{3^5}\approx\boxed2$.\\
\\
(5): $\int_0^2(x^2+3x-1)dx\\
=[\frac{1}{3}x^3+\frac{3}{2}x^2-x]_0^2\\
=\frac{8}{3}+6-2\\
=\frac{20}{3}\\
\approx\boxed7$
\end{document}