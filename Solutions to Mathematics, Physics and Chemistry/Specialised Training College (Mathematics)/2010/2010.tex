\documentclass{article}

\usepackage{amsmath}
\usepackage{amssymb}

\linespread{2}

\begin{document}
\noindent Q1:\\
Put $(x,y)=(-2,41)$ and $(x,y)=(5,20)$ into the functions respectively, we have equations:\\
$4A+2B+C=41......(1)$\\
$25A-5B+C=20......(2)$.\\
On the other hand, by completing the square, we have $y=A(x-\frac{B}{2A})^2-\frac{B^2}{4A}+C$.\\
As the function is minimized at $x=2$, we have $\frac{B}{2A}=2$, i.e. $4A-B=0......(3)$.\\
(2)-(1): $21A-7B=-21$, i.e. $3A-B=-3......(4)$.\\
(3)-(4): $A=\boxed{3}$.\\
Substitue $A=3$ into $(3)$, $B=4A=\boxed{12}$.\\
Substitue $(A,B)=(3,12)$ into $(1)$, $C=41-4A-2B=41-12-24=\boxed5$.\\
Moreover, when $x=2$, we obtain the minimum value of the function, which is $y=-\frac{B^2}{4A}+C=-\frac{12^2}{12}+5=-12+5=\boxed{-7}$.\\
(Note: The minimum value can also be calculated by putting $x=2$ into the function.)\\
\\
\textbf{Alternative (with calculus)}  The equation (3) can also be obtained as the following:\\
$y^\prime=2Ax-B$.\\
When $y$ attains to its extremum, $y^\prime=0$. Hence, by putting $x=2$, we have $4A-B=0......(3)$.

\vspace{1cm}
\hrule
\vspace{1cm}

\noindent Q2:\\
As $x$ satisfies $x^2+2x-2=0$, we have $x^3=-2x^2+2x$. Therefore, $P$ can be rewritten as $P=(-2x^2+2x)+x^2+ax+1=-x^2+(a+2)x+1$.\\
Moreover, as $x^2=-2x+2$, $P=-(-2x+2)+(a+2)x+1=(a+4)x-1.$\\
As $P$ is independent on the value of $x$, we have $a+4=0$, i.e. $a=\boxed{-4}$.\\
In this case, the value of $P$ is $\boxed{-1}$.

\vspace{1cm}
\hrule
\vspace{1cm}

\noindent Q3:\\
(i): $$x^2-3x-10<0$$
$$\iff (x-5)(x+2)<0$$
$$\iff \boxed{-2<x<5}$$
(ii): $$|x-2|<a$$
$$\iff -a<x-2<a$$
$$\iff 2-a<x<a+2$$
In case $x^2-3x-10<0\implies |x-2|<a$, we have $$2-a\leq-2\text{ and }5\leq a+2$$
$$\iff 4\leq a\text{ and }3\leq a$$
$$\iff \boxed{a\geq 4}$$
(iii) In case $|x-2|<a\implies x^2-3x-10<0$, we have $$-2\leq 2-a\text{ and }a+2\leq 5$$
$$\iff a\leq 4\text{ and }a\leq 3$$
$$\iff a\leq 3$$
On the other hand, for the inequality $|x-2|<a$ holds with a solution, we have the hidden condition $a>0$.\\
Combine the above, we have $\boxed{0<a\leq3}$.

\vspace{1cm}
\hrule
\vspace{1cm}

\noindent Q4:\\
(1):  $X=x\leq4$ if and only if $x\in\{1,2,3,4\}$, where the set has 4 elements.\\
On the other hand, the universal set, $\{1,2,3,4,5,6\}$, has 6 elements.\\
Therefore, $P(X=x\leq 4)=\frac{4}{6}=\frac{2}{3}$.\\
Then, $P(B)=(P(X=x\leq4))^3=(\frac{2}{3})^3=\boxed{\frac{8}{27}}.$\\
Similarly, $P(C)=(P(X=x\leq3))^3=(\frac{3}{6})^3=\boxed{\frac{1}{8}}.$\\
(2): $P(B)=P(A\cup C)\iff P(B)=P(A)+P(C)$ as $A$ and $C$ are mutually exclusive.\\
Therefore, $P(A)=P(B)-P(C)=\frac{8}{27}-\frac{1}{8}=\boxed{\frac{37}{216}}.$

\vspace{1cm}
\hrule
\vspace{1cm}

\noindent Q5:\\
Square the first equality, we have \\$(x+y+z)^2=x^2+y^2+z^2+2(xy+yz+zx)=3^2=9.$\\
By the second equality, we have $x^2+y^2+z^2=9$.\\
Substitue it into the equality we got, we have $xy+yz+zx=\boxed0$.\\
Next, $(x^2+y^2+z^2)^2=x^4+y^4+z^2+\boxed{2}((xy)^2+(yz)^2+(zx)^2)$\\
$=x^4+y^4+z^2+2((xy+yz+zx)^2-2(xy^2z+yz^2x+zx^2y))\\
=x^4+y^4+z^2+2((0)^2-2xyz(x+y+z))\\
=x^4+y^4+z^2+2(-2(-2)(3))\\
=x^4+y^4+z^2+24=(9)^2=81.$\\
Therefore, $x^4+y^4+z^2=81-24=\boxed{57}.$

\vspace{1cm}
\hrule
\vspace{1cm}

\noindent Q6:\\
(1): We have \\
$\triangle ADF=\frac{1}{2}(AD)(AF)\sin\angle A=\frac{1}{2}(AD)(DF)\sin60^\circ=\frac{\sqrt3}{4}AD\cdot DF.$\\
On the other hand, by the circle power theorem, we have $AD\cdot DF=AG\cdot AE$.\\
Therefore, $\frac{\triangle ADF}{AG\cdot AE}=\frac{\frac{\sqrt3}{4}AG\cdot AE}{AG\cdot AE}=\boxed{\frac{\sqrt3}{4}}.$\\
(2): As both $BE,BD$ and $CE,CF$ are tangent to the circle, we have \\$BE=BD=4$ and $CE=CF=2$.\\
Then, $BC=BE+EC=4+2=\boxed6$.\\
On the other hand, $AF=AD=x$. By the cosine formula, we have 
$$BC^2=AB^2+AC^2-2(AB)(AC)\cos\angle A$$
$$6^2=(x+4)^2+(x+2)^2-2(x+4)(x+2)\cos 60^\circ$$
$$36=x^2+8x+16+x^2+4x+4-x^2-6x-8$$
$$x^2+\boxed6x-\boxed{24}=0$$
$$x=\frac{-6\pm\sqrt{6^2-4(1)(-24)}}{2}=-3\pm\sqrt{33}$$
As $AD>0$, we have $AD=\boxed{-3+\sqrt{33}}$.

\vspace{1cm}
\hrule
\vspace{1cm}

\noindent Q7:\\
If the x-coordinate of $P$ is $\alpha$, then the x-coordinates of $Q$ and $T$ will also be $\alpha$. Moreover, by symmetry along the y-axis (as the axis of symmetry of the parabolas is $x=0$), the x-coordinate of $R$ will be $-\alpha$.\\
As $Q,R$ lie on a straight line parallel to the x-axis, we have \\
$QR=\alpha-(-\alpha)=\boxed2\alpha$.\\
Moreover, substitue $x=\alpha$ into the equations of the two parabolas respectively, the y-coordinates of $Q$ and $T$ are $-\alpha^2+4$ and $\frac{1}{2}\alpha^2-2$ respectively.\\
As $Q,P,T$ lie on a straight line parallel to the y-axis, we have\\
$PQ=(-\alpha^2+4)-0=\boxed4-\alpha^2$ and $PT=0-(\frac{1}{2}\alpha^2-2)=\boxed2-\frac{1}{\boxed2}\alpha^2$.\\
Then, we have $l=2(QR+RT)=2(2\alpha+(4-\alpha^2+2-\frac{1}{2}\alpha^2))=\boxed{12}+\boxed4\alpha-\boxed3\alpha^2$.\\
By completing the square, we have $l=-3(\alpha-\frac{2}{3})^2+\frac{40}{3}$.\\
As $-2(\alpha-\frac{2}{3})^2\leq0$ and the equality holds when $\alpha=\frac{2}{3}$, we have $l\leq \frac{40}{3}$.\\
When $\alpha=\boxed{\frac{2}{3}}$, $l$ is maximized and the value of it is $\boxed{\frac{40}{3}}$.

\end{document}
