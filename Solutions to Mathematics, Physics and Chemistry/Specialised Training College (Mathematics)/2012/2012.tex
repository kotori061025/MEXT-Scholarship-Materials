\documentclass{article}
\linespread{2}
\usepackage{amsmath}

\begin{document}
\noindent Q1(1):\\
$(2x+1)^2-4(x^2-3)\\
=(4x^2+4x+1)-(4x^2-12)\\
=\boxed{4x+13}$

\vspace{1cm}
\hrule
\vspace{1cm}

\noindent Q1(2):\\
$(-2x^2y)^3\div(xy^2z)^2\times(-yz)^2\\
=\frac{(-8x^6y^3)(y^2z^2)}{x^2y^4z^2}\\
=\boxed{-8x^4y}$

\vspace{1cm}
\hrule
\vspace{1cm}

\noindent Q1(3):\\
As $3$ is a solution, we have $9+3p+12=0$, i.e. $p=\boxed{-7}$.\\
Solving $$x^2-7x+12=0$$
$$(x-3)(x-4)=0$$
$$x=3,\boxed4$$

\vspace{1cm}
\hrule
\vspace{1cm}

\noindent Q1(4):\\
$x^2+y^2=(x+y)^2-2xy=9-2=\boxed7$.\\
$x^3+y^3=(x+y)^3-3xy(x+y)=27-3(1)(3)=18$.\\
$x^4+y^4=(x^2+y^2)^2-2(xy)^2=49-2=47$.\\
$x^5+y^5=(x+y)(x^4+y^4)-xy(x^3+y^3)=(3)(47)-(1)(18)=\boxed{123}$.\\
\\
Note: There are many ways to solve this question, but the main idea is the same: figure out a way to rewrite the symmetric polynomials in terms of elementary symmetric polynomials.

\vspace{1cm}
\hrule
\vspace{1cm}

\noindent Q1(5):\\
Note that $\sqrt[4]{4}=\sqrt{2}$.\\
To compare $\sqrt[3]{3}$ with $\sqrt2$, we take the cube of each:\\
$\sqrt[3]{3}^3=3$\\
$\sqrt{2}^3=2\sqrt2\approx2\cdot1.4=2.8<3$.\\
Therefore, $\boxed{\sqrt[3]{3}}$ is the largest.\\
On the other hand, $\cos45^\circ=\frac{\sqrt2}{2}$, $\tan30^\circ=\frac{\sqrt3}{3}$, $\sin120^\circ=\frac{\sqrt3}{2}$.\\
Comparing their squares: $\frac{1}{3}<\frac{1}{2}<\frac{3}{4}$.\\
Therefore, $\tan30^\circ$ is the smallest.

\vspace{1cm}
\hrule
\vspace{1cm}

\noindent Q1(6):\\
Choosing 3 points among the 6 point to make a triangle, the total number of ways=$C_3^6=\boxed{20}$.

\vspace{1cm}
\hrule
\vspace{1cm}

\noindent Q1(7):\\
The progression is an arithmetic progression with first term 1 and common difference $3$. Therefore, the general term is $1+(n-1)(3)$.\\
The 20th term=$1+3\cdot19=\boxed{58}$.\\
The sum=$\frac{(1+58)(20)}{2}=\boxed{590}$.

\vspace{1cm}
\hrule
\vspace{1cm}

\noindent Q1(8):\\
The area of the triangle=$\frac{1}{2}(2)(2)\sin60^\circ=\sqrt3$.\\
Consider the sum of areas of each triangles divided by the angle bisectors:
$$3(\frac{1}{2}(2)(r))=\sqrt3$$
$$r=\boxed{\frac{\sqrt3}{3}}$$
\\
Note: There are lots of ways to solve this question. However, for finding inradius, considering the areas will be the general methodology. Other methods are not likely to be simplier than this method.

\vspace{1cm}
\hrule
\vspace{1cm}

\noindent Q1(9):\\
As $\frac{dy}{dx}=6x-4$, the slope of the tangent=$\frac{dy}{dx}|_{x=1}=2$.\\
Using the point-slope form of striaght line, the equation is $y=2(x-1)=\boxed{2x-2}$.\\
The area=$\int_0^1((3x^2-4x+1)-(2x-2))dx\\
=[x^3-3x^2+3x]_0^1\\
=1-3+3\\
=\boxed1$

\vspace{1cm}
\hrule
\vspace{1cm}

\noindent Q2:\\
(1): Note that the slope of BA multiplying the slope of BC=$(\frac{4-2}{3-2})(\frac{0-2}{6-2})=-1$.\\
Therefore, $BA\perp BC$ and hence $AC$ is the diameter.\\
The centre is the mid-point of $AC$, i.e. $\boxed{(\frac{9}{2},2)}$.\\
Radius=$\frac{AC}{2}=\frac{\sqrt{(3-6)^2+(4-0)^2}}{2}=\boxed{\frac{5}{2}}$.\\
\\
(2): As shown in (1), $BA\perp BC$. Therefore, $\vec{BA}\cdot\vec{BC}=\boxed0$.\\
\\
(3): As shown, $\angle ABC=\boxed{90^\circ}$.\\
\\
(4): The equation of AC is $y=\frac{4-0}{3-6}(x-6)$, i.e. $4x+3y-24=0$.\\
Therefore, the distance=$\frac{|4(2)+3(2)-24|}{\sqrt{4^2+3^2}}=\frac{10}{5}=\boxed{2}$.\\
\\
(5): By the angle bisector theorem, $AL:LC=AB:CB=\sqrt{(3-2)^2+(4-2)^2}:\sqrt{(6-2)^2+(0-2)^2}=1:\boxed2$.\\
\\
(6): As $AL:LC=1:2$, we have $\vec{BL}=\frac{1}{1+2}(2\vec{BA}+\vec{BC})=\boxed{\frac{2}{3}}\vec{BA}+\boxed{\frac{1}{3}}\vec{BC}$.

\vspace{1cm}
\hrule
\vspace{1cm}

\noindent Q3:\\
(1): As the parabola is convex upwards, we have $a\boxed{<}0$.\\
\\
(2): By completing the square, we have $y=a(x+\frac{b}{2a})^2+c-\frac{b^2}{4a}$. Therefore, the axis of symmetry of the parabola is $x=-\frac{b}{2a}$. From the graph, we have $-\frac{b}{2a}>0$. As $a<0$, we have $b\boxed{>}0$.\\
\\
(3): As the slope of the line is positive, we have $d\boxed>0$.\\
\\
(4): As the parabola has two x-intercepts, we have $\Delta=b^2-4ac\boxed>0$.\\
\\
(5): When $x=1$, the parabola intersects the x-axis. Therefore, we have $a+b+c\boxed=0$.\\
\\
(6): When $x=-1$, the parabola takes negative y value. Therefore, we have $a-b+c\boxed<0$.\\
\\
(7): When $x=2$, the parabola takes negative y value. Therefore, we have $4a+2b+c\boxed<0$.\\
\\
(8): As the y-intercept of the parabola (c) is positive while that for the line (e) is negative, we have $c-e\boxed{>}0$.
\end{document}